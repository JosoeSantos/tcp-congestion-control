
\documentclass[12pt,a4paper]{article}

% Pacotes básicos
\usepackage[utf8]{inputenc}
\usepackage[T1]{fontenc}
\usepackage[brazil]{babel}
\usepackage{graphicx}
\usepackage{float}
\usepackage{amsmath, amssymb}
\usepackage{hyperref}
\usepackage{caption}
\usepackage{listings} % para formatar blocos de código
\usepackage{enumitem} % para controlar listas
\usepackage{xcolor}   % necessário para cores no listings
\usepackage{multirow}
\usepackage{booktabs}
\usepackage{biblatex}
\usepackage{csquotes}
\usepackage{subfigure}


% Configurações do listings
\lstset{
  language=Python,
  basicstyle=\ttfamily\small,
  numbers=left,
  numberstyle=\tiny,
  frame=single,
  breaklines=true,
  keywordstyle=\color{blue}\bfseries,
  stringstyle=\color{red},
  commentstyle=\color{green!60!black}\itshape,
  showstringspaces=false
}

\graphicspath{{./img/}}
\addbibresource{references.bib}


\begin{document}

% ==============================
% CAPA
% ==============================
\begin{titlepage}
    \centering
    {\Large \textbf{Universidade Federal de Minas Gerais}}\\[0.3cm]
    {\large Engenharia de Sistemas}\\[2cm]
    
    {\Huge \textbf{Analise de Controle de Congestionamento TCP}}\\[1.5cm]
    
    \textbf{Introdução a Sistemas Distribuidos}\\[0.5cm]
    \textbf{Professor: Aldri Luiz dos Santos }  \\[1.5cm]
    
    \begin{flushleft}
        \textbf{Aluno:}\\
        Josoé Santos Queiroz --- 2019026982
    \end{flushleft}
    
    \vfill
    {\large Belo Horizonte, MG}\\
    {\large \today}
\end{titlepage}

\clearpage
\tableofcontents
\clearpage

% ==============================
% INTRODUÇÃO
% ==============================
\section{Introdução}

O protocolo TCP (Transmission Control Protocol) é um dos principais protocolos de comunicação utilizados na internet para garantir a entrega confiável de dados entre dispositivos. Um dos desafios enfrentados pelo TCP é o controle de congestionamento, que visa evitar a sobrecarga da rede e garantir um desempenho eficiente na transmissão de dados. 

Neste trabalho, exploramos diferentes condições para o controle de congestionamento em um link de gargalo. Analisamos também o impacto que diferentes algoritmos de controle de congestionamento têm sobre o desempenho da rede, utilizando simulações do software NS3 \cite{nsnam_2019_ns3} para avaliar métricas como taxa de transferência, atraso e perda de pacotes.

\section{Metodologia}

O experimento foi condizido utilizando duas topologias de rede distintas, cada uma representando diferentes condições de controle de congestionamento. A primeira topologia ilustrada pela Figura \ref{fig:topologia1} consiste em um cenário de rede aonde só existe uma linha entre as duas pontas extremas da topologia. Já a segunda topologia, ilustrada pela Figura \ref{fig:topologia2}, consiste em um cenário de rede aonde ao final do link de gargalo, existem duas linhas diferentes que podem ser utilizadas para o envio dos pacotes.

\begin{figure}[H]
    \centering
    \includegraphics[width=0.6\textwidth]{./img/topologia1.png}
    \caption{Topologia 1: Rede com um único caminho entre as pontas extremas.}
    \label{fig:topologia1}
\end{figure}

\begin{figure}[H]
    \centering
    \includegraphics[width=0.6\textwidth]{./img/topologia2.png}
    \caption{Topologia 2: Rede com dois caminhos possíveis entre as pontas extremas.}
    \label{fig:topologia2}
\end{figure}

Para a primeira topologia observamos a diferença de desempenho entre os algorítmos de controle de congestionamento TCP NewReno e TCP CUBIC \cite{ha_cubic}. Para isto foram realizados testes variando a o delay do link de gargalo e a taxa de erro de pacotes.

%% Paragrafo gerado por IA REVISAR
Já para a segunda topologia, além de observarmos a diferença de desempenho entre os algorítmos de controle de congestionamento TCP NewReno e TCP CUBIC, também observamos o impacto do balanceamento de carga na rede. Para isto foram realizados testes variando a o delay do link de gargalo, a taxa de erro de pacotes e o balanceamento de carga entre os dois links disponíveis.

A lista de experimentos realizados está descrita na Tabela \ref{tab:experimentos}.

\begin{table}[H]
    \centering
    \begin{tabular}{c|c|c|c|c}
        \toprule
        Teste &  Delay (ms) & D.R. (Mbps) & Taxa de Erro (\%) & Fluxos TCP \\
        \bottomrule
        \toprule
        \multirow{1}{*}{1.a} & 100 & 10 & 1e-05 & 1  \\        \hline
        \multirow{6}{*}{1.b} & 50 & 1 & 1e-05 & 1,2,4 \\
                              & 100 & 1 & 1e-05 & 1,2,4 \\
                              & 150 & 1 & 1e-05 & 1,2,4 \\
                              & 200 & 1 & 1e-05 & 1,2,4 \\
                              & 250 & 1 & 1e-05 & 1,2,4 \\
                              & 300 & 1 & 1e-05 & 1,2,4 \\
        \hline
        \multirow{6}{*}{1.c} & 100 & 1 & 1e-05 & 1,2,4 \\
                              & 100 & 1 & 5e-05 & 1,2,4 \\
                              & 100 & 1 & 1e-04 & 1,2,4 \\
                              & 100 & 1 & 5e-04 & 1,2,4 \\
                              & 100 & 1 & 1e-03 & 1,2,4 \\
        \hline
        2 & 100 & 1 & 1e-05 & 2,4,6,8 \\
        \bottomrule
        \hline
    \end{tabular}
    \caption{Lista de experimentos realizados.}
    \label{tab:experimentos}

\subsection{Ambiente de testes}

Os experimentos foram realizados utilizando o simulador de redes NS3 (Network Simulator 3) \cite{nsnam_2019_ns3} versão 3.46.1, em um ambiente MacOS Tahoe com um processador Apple M4 e 16GB de memória RAM. As simulações foram configuradas para rodar por um período de 20 segundos e os fluxos TCP foram iniciados no segundo 1 da simulação conforme especificado pelo professor. 

A fim de simplificar o fluxo de ajustes e execução dos experimentos, foram desenvolvidos scripts em bash para automatizar o processo de configuração, execução e coleta de dados das simulações. Esses scripts permitiram a variação sistemática dos parâmetros de rede, como delay, taxa de erro e número de fluxos TCP, conforme descrito na Tabela \ref{tab:experimentos}. Além disso os mesmos scripts configuraram uma execução paralela das similações, otimizando o tempo total necessário para a realização dos experimentos. O códigoo fonte das simulações, dos scripts de automação, dos plots e deste relatório estão disponíveis no repositório GitHub \cite{gh_code}.

\end{table}

\section{Resultados}

O experimento 1.a foi condizido para observar a curva de crescimento da janela de congestionamento para os algorítmos TCP NewReno e TCP CUBIC. A Figura \ref{fig:exp1a1} apresenta o resultado obtido. A comparação da Congestion Window (Cwnd) feita demonstra que o TCP CUBIC apresenta um crescimento mais estável e permite um fluxo de dados mais consistente em comparação ao TCP NewReno, que exibe variações mais abruptas na janela de congestionamento.

\begin{figure}
    \centering
    \hfill
    \subfigure[Cwnd CUBIC]{\includegraphics[width=0.4\textwidth]{./img/cwnd-TcpCubic.png}}
    \hfill
    \subfigure[Cwnd NewReno]{\includegraphics[width=0.4\textwidth]{./img/cwnd-TcpNewReno.png}}    \hfill
    \caption{Comparação do crescimento da janela de congestionamento para TCP CUBIC e TCP NewReno no experimento 1.a.}
    \label{fig:exp1a1}
\end{figure}

Para o segundo experimento 1.b (ver tabela \ref{tab:experimentos}), foram realizados testes variando o delay do link de gargalo. A Figura \ref{fig:exp1b} apresenta o resultado obtido. Observa-se que o TCP CUBIC mantém uma taxa de transferência mais alta e estável em comparação ao TCP NewReno indicando melhor adaptação a variações de delay na rede.

\begin{figure}
    \centering
    \includegraphics[width=0.7\textwidth]{img/goodput_vs_delay.png}
    \caption{Comparação do throughput para TCP CUBIC e TCP NewReno no experimento 1.b.}
    \label{fig:exp1b}
\end{figure}

O terceiro experimento 1.c, foram realizados testes variando a taxa de erro de pacotes. A Figura \ref{fig:exp1c} apresenta o resultado obtido. Observa-se que o TCP CUBIC mantém uma taxa de transferência mais alta e estável em comparação ao TCP NewReno indicando melhor adaptação a variações na taxa de erro de pacotes na rede. Para melhor visualização dos resultados, os gráficos foram plotados em escala logarítmica e linear.
\begin{figure}
    \centering
    \hfill
    \subfigure[Escala Logarítmica]{\includegraphics[width=0.45\textwidth]{./img/goodput_vs_error_rate_log.png}}
    \hfill
    \subfigure[Escala Linear]{\includegraphics[width=0.45\textwidth]{./img/goodput_vs_error_rate_linear.png}}    \hfill
    \caption{Comparação do throughput para TCP CUBIC e TCP NewReno no experimento 1.c.}
    \label{fig:exp1c}
\end{figure}

Os experimentos 1.a, 1.b e 1.c demonstram que o TCP CUBIC é mais eficiente em manter uma taxa de transferência elevada e estável em condições adversas de rede, como altos delays e taxas de erro elevadas, em comparação ao TCP NewReno. Isso está em conformidade com \cite{ha_cubic} que em seu trabalho afirma a eficiência do TCP cúbic em ambientes de rede modernos.

Por fim o trabalho também analisou o impacto do balanceamento de carga na rede utilizando a topologia 2. A Figura \ref{fig:exp2} apresenta o resultado obtido. Observa-se que o balanceamento de carga entre os dois links disponíveis melhora significativamente a taxa de transferência para ambos os algoritmos de controle de congestionamento, com o TCP CUBIC continuando a apresentar um desempenho superior ao TCP NewReno.

\begin{figure}
    \centering
    \includegraphics[width=0.7\textwidth]{img/goodput_vs_flows.png}
    \caption{Comparação do throughput para TCP CUBIC e TCP NewReno no experimento 2 com balanceamento de carga.}
    \label{fig:exp2}
\end{figure}

\section{Conclusões}

Este trabalho explorou o desempenho de diferentes algoritmos de controle de congestionamento TCP, especificamente TCP NewReno e TCP CUBIC. Através de simulações utilizando o NS3, foi possível observar que o TCP CUBIC apresenta um desempenho superior em termos de taxa de transferência, especialmente em condições adversas de rede, como altos delays e taxas de erro elevadas. Além disso, o balanceamento de carga mostrou-se eficaz na melhoria do desempenho da rede para ambos os algoritmos.

Além disso, o trabalho desenvolvido proporcionou uma compreensão mais profunda dos mecanismos de controle de congestionamento e das dinâmicas de teste e valicação de diferentes estratégias. O software NS3 mostrou-se uma ferramenta valiosa para a simulação e análise de redes, permitindo a avaliação detalhada do desempenho dos algoritmos em diferentes cenários. A possibilidade de simular uma rede com uma complexidade considerável e obter métricas com relativa facilidade foi crucial para o sucesso deste estudo.

\section{Referências}

\printbibliography 

\clearpage

\end{document}
